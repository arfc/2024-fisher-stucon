\documentclass{anstrans}
%%%%%%%%%%%%%%%%%%%%%%%%%%%%%%%%%%%
\title{Tribal Sovereignty and Nuclear Power}
\author{Riley J. Fisher,$^{*}$ Samuel G. Dotson, $^{*}$ Madicken Munk,$^{*}$}

\institute{ $^{*}$Dept. of Nuclear, Plasma, and Radiological Engineering,
University of Illinois at Urbana-Champaign, Urbana, IL 61801 }

% % Optional disclaimer: remove this command to hide \disclaimer{Notice: this
% manuscript is a work of fiction. Any resemblance to actual articles, living or
% dead, is purely coincidental.}

%%%% packages and definitions (optional)
\usepackage{graphicx} % allows inclusion of graphics
\usepackage{booktabs} % nice rules (thick lines) for tables
\usepackage{microtype} % improves typography for PDF
% \usepackage{xcolor}

\newcommand{\SN}{S$_N$}
\renewcommand{\vec}[1]{\bm{#1}} %vector is bold italic
\newcommand{\vd}{\bm{\cdot}} % slightly bold vector dot
\newcommand{\grad}{\vec{\nabla}} % gradient
\newcommand{\ud}{\mathop{}\!\mathrm{d}} % upright derivative symbol

\begin{document}
%%%%%%%%%%%%%%%%%%%%%%%%%%%%%%%%%%%%%%%%%%%%%%%%%%%%%%%%%%%%%%%%%%%%%%%%%%%%%%%%


\section{Introduction}
The majority of research relating nuclear energy and native tribes has focused
on reparative efforts. Examples include studies of mismanaged mining operations
\cite{hoover_elevated_2017}, failed waste management projects
\cite{endres_sacred_2012}, and hazardous weapons testing programs
\cite{frohmberg_assessment_2000} --- all harmful activities. While valuable for
learning from mistakes, understanding oversights, and developing means to
alleviate these harms, this area of study focuses solely on how nuclear energy
has harmed indigenous communities. This paper shifts from the perspectives of
previous studies and instead explores the capabilities of nuclear energy to
uplift tribal communities. This paper will provide a brief overview of energy
sovereignty in native land, followed by a preliminary investigation into a
potential future that would utilize nuclear energy to achieve energy goals. The
results of this study will be presented at the 2024 American Nuclear Society
Student Conference. 

\subsection{Energy Sovereignty}
Although different groups and governments define energy sovereignty in myriad
ways, this work utilizes the definition provided by Laldjebaev et al. 2016
\cite{laldjebaev_energy_2016}: 
\begin{quote}
  [Energy sovereignty is] a framework that recognizes the individual, community,
  or nation's rights, and strengthens their abilities to exercise choice within
  all components of energy systems, including sources, means of harnessing, and
  uses in order to satisfy their needs for energy.
\end{quote}
While part of a larger overall discussion of tribal sovereignty, the concept of
energy sovereignty for tribal nations is related to the dependency on power
produced by the United States. Achieving energy sovereignty and independence
from the U.S. electric grid is a primary goal of many tribal energy initiatives
\cite{western_area_power_administration_tribal_2010}, which is often pursued
through Tribal Energy Resource Agreements (TERAs) and Tribal Energy Development
Organizations (TEDOs) \cite{department_of_interior_25_2008}. Additionally, as
outlined in United States legislation, the Department of Energy (DOE) has a
responsibility to aid in the energy goals of native tribes. The Energy Policy
Act of 1992 \cite{rep_sharp_hr776_1992} states "The Secretary of Energy… shall
establish and implement a demonstration program to assist in Indian tribes in
pursuing energy self-sufficiency and to promote the development of a vertically
integrated industry on Indian reservations […]" by providing grants,
low-interest loans, and technical assistance. Recently, DOE reaffirmed its right
to preferentially purchase electricity from tribal groups with the Indian Energy
Purchase Preference (IEPP) \cite{granholm_memorandum_2023}. The mutual benefit
among native tribes and the U.S. federal government is a key motivator for both
groups prioritizing tribal energy sovereignty. Within the preliminary
investigation and literature review, no energy-producing tribal nation was
identified to be entirely energy sovereign. Furthermore, despite the
reaffirmation of the IEPP, no use of this authority has been identified. The
next section outliness the challenges to tribal nations' energy sovereignty.

\subsection{Barriers to Independence}
An extensive review of the barriers facing tribal energy sovereignty was
published in early 2024 \cite{raimi_securing_2024}. The `key barriers to tribal
energy sovereignty' proposed in this paper can be categorized into two factors:
Historical (contexts that cannot be changed) and modern (challenges that could,
in theory, be changed). The barriers titled `Ethnic Cleansing,' `Forced
Migration,' `Forced Coexistence,' and `Land Fragmentation,' are examples of
historical factors. These are challenges that altered the identities of native
communities as a result of U.S. government actions. The barriers of `Inadequate
Consultation,' `Federal Bureaucracy,' `Lack of Institutional Capacity,' and
`Inability to access tax credits,' are examples of modern factors that could be
addressed with proper intervention \cite{raimi_securing_2024}. Economic,
regulatory, and political factors constitute the main modern barriers to tribal
energy sovereignty. An important nuance to the complexity of regulatory
solutions to energy sovereignty is highlighted by Former Senator Ben Campbell
\cite{kronk2012} : 
\begin{quote}
  The Committee on Indian Affairs has been informed over the year that the
  Secretarial approval process is often so lengthy that outside parties, who
  otherwise would like to partner with Indian tribes to develop their energy
  resources are reluctant to become entangled in the bureaucratic red tape that
  inevitably accompanies the leasing of Tribal resources.
\end{quote}
This quote, combined with previous efforts to enhance tribal energy sovereignty
described above, illustrate the ways government regulation could help or hinder
tribal goals. This complex interaction of regulation, economics, and technology,
is a major reason no tribe has achieved complete energy sovereignty.

%%%%%%%%%%%%%%%%%%%%%%%%%%%%%%%%%%%%%%%%%%%%%%%%%%%%%%%%%%%%%%%%%%%%%%%%%%%%%%%%
\section{Preliminary Results}
The U.S. Department of Energy's Office of Nuclear Energy Nuclear Energy Tribal
Working Group (NETWG) leads the endeavor of using nuclear energy to advance
tribal nations' energy objectives. This group, which includes representatives
from 12 tribes and tribal organizations, is dedicated to expanding educational
and economic opportunities, management of spent fuel, emergency preparedness,
and advancement of emergent reactor technologies for tribal groups
\cite{department_of_energy_nuclear_2023}. Economic development and technological
advancement are directly related to energy sovereignty. While the other
initiatives may benefit overall goals of self-sufficiency, they will not be a
core focus of this work. The NETWG has two publicly available documents that
report on the status of energy-related U.S.-tribal relations. The first,
published two years after the NETWG charter, discusses the history of
consent-based siting on tribal land
\cite{nuclear_energy_tribal_working_group_2016}, while the second reports on the
status of educational opportunities in native lands
\cite{nuclear_energy_tribal_working_group_2016}. Additionally, the NETWG
announced a \$1.5 million funding opportunity in 2023 facilitate increased
communication between tribal and federal nuclear energy representatives. Despite
limited public awareness regarding in-depth government efforts and initiatives,
it is evident that there are groups interested in and resources available for
further work pertaining to nuclear-based tribal development.

\subsection{Further Investigations}
Due to the significant knowledge and development gaps surrounding this area of
research, this study aims to serve as initial grounds for further exploration
into nuclear energy solutions for native nations. \textbf{The authors intend to
survey tribal communities to determine the interest in tribal-owned nuclear
power, as well as interview tribal leaders to strengthen context surrounding
economic, regulatory, and cultural (e.g., education, risk-perception)
challenges.} Additionally, this study will include a review of existing federal
energy and tribal legislation and nuclear power regulation. These reviews will
determine the extent of regulations regarding nuclear technology deployment
within tribal nations and may result in recommendations for legal or regulatory
changes. Subsequently, the authors will demonstrate a feasibility utilizing a
tribal-owned microreactor deployment scenario. The progress and findings of this
work will be presented at the 2024 American Nuclear Society Student Conference.


%%%%%%%%%%%%%%%%%%%%%%%%%%%%%%%%%%%%%%%%%%%%%%%%%%%%%%%%%%%%%%%%%%%%%%%%%%%%%%%%


%%%%%%%%%%%%%%%%%%%%%%%%%%%%%%%%%%%%%%%%%%%%%%%%%%%%%%%%%%%%%%%%%%%%%%%%%%%%%%%%
% \section{Acknowledgments}

%%%%%%%%%%%%%%%%%%%%%%%%%%%%%%%%%%%%%%%%%%%%%%%%%%%%%%%%%%%%%%%%%%%%%%%%%%%%%%%%
\bibliographystyle{ans}
\bibliography{2024-fisher-stucon}
\end{document}

