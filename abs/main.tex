\documentclass{anstrans}
%%%%%%%%%%%%%%%%%%%%%%%%%%%%%%%%%%%
\title{Tribal Sovereignty and Nuclear Power}
\author{Riley J. Fisher,$^{*}$ Samuel G. Dotson, $^{*}$ Madicken Munk,$^{*}$}

\institute{ $^{*}$Dept. of Nuclear, Plasma, and Radiological Engineering,
University of Illinois at Urbana-Champaign, Urbana, IL 61801 }

% Optional disclaimer: remove this command to hide
\disclaimer{Notice: this manuscript is a work of fiction. Any resemblance to
actual articles, living or dead, is purely coincidental.}

%%%% packages and definitions (optional)
\usepackage{graphicx} % allows inclusion of graphics
\usepackage{booktabs} % nice rules (thick lines) for tables
\usepackage{microtype} % improves typography for PDF

\newcommand{\SN}{S$_N$}
\renewcommand{\vec}[1]{\bm{#1}} %vector is bold italic
\newcommand{\vd}{\bm{\cdot}} % slightly bold vector dot
\newcommand{\grad}{\vec{\nabla}} % gradient
\newcommand{\ud}{\mathop{}\!\mathrm{d}} % upright derivative symbol

\begin{document}
%%%%%%%%%%%%%%%%%%%%%%%%%%%%%%%%%%%%%%%%%%%%%%%%%%%%%%%%%%%%%%%%%%%%%%%%%%%%%%%%


\section{Introduction}
The majority of research that has been conducted regarding nuclear energy and
native tribes has been largely focused on reparative efforts. Infamous examples
include studies of mismanaged mining operations \cite{hoover_elevated_2017},
failed waste management projects \cite{endres_sacred_2012}, and hazardous
weapons testing programs \cite{frohmberg_assessment_2000} – all activities that
have caused significant harm in the past. While incredibly valuable for learning
from mistakes and oversights and developing means to alleviate these harms, this
area of study focuses solely on how nuclear energy has harmed communities. This
paper shifts from the perspectives of previous studies and aims instead to
explore the capabilities of nuclear energy to uplift tribal communities. This
paper will provide a brief overview of energy sovereignty in native land,
followed by a preliminary investigation into a potential future that would
utilize nuclear energy to achieve energy goals. The final findings of this study
will be presented at the 2024 American Nuclear Society Student Conference. 

\subsection{Energy Sovereignty}
While there is no consolidated definition of energy sovereignty by any group or
government, many definitions have been proposed. This work will utilize the
definition provided in \cite{laldjebaev_energy_2016}, as “… a framework that
recognizes the individual, community, or nation's rights, and strengthens their
abilities to exercise choice within all components of energy systems, including
sources, means of harnessing, and uses in order to satisfy their needs for
energy.” While part of a much larger overall discussion of tribal sovereignty,
the concept of energy sovereignty for tribal nations is related to the
dependency of United States-produced power. A main goal of many tribal energy
initiatives is to achieve energy sovereignty and independence from the U.S.
electric grid \cite{western_area_power_administration_tribal_2010}, which is
often pursued through Tribal Energy Resource Agreements (TERAs) and Tribal
Energy Development Organizations (TEDOs) \cite{department_of_interior_25_2008}.
Additionally, as outlined in United States legislation, the Department of Energy
has a responsibility to aid in the energy goals of native tribes. The Energy
Policy Act of 1992 \cite{rep_sharp_hr776_1992} states “The Secretary of Energy…
shall establish and implement a demonstration program to assist in Indian tribes
in pursuing energy self-sufficiency and to promote the development of a
vertically integrated industry on Indian reservations…” by providing grants,
low-interest loans, and technical assistance. Recently, the Department of Energy
has reaffirmed its right to preferentially purchase electricity from tribal
groups with the Indian Energy Purchase Preference (IEPP)
\cite{granholm_memorandum_2023}. The mutual benefit among native tribes and the
U.S. federal government is a key reason as to why tribal energy sovereignty
remains a high priority for both groups. Within the preliminary investigation
and literature review, no energy-producing tribal nation was identified to be
entirely energy sovereign. Furthermore, despite the reaffirmation of the IEPP,
no use of this authority has been identified. While important to exercise
patience in these extensive transition endeavors, it is necessary to determine
the challenges to energy sovereignty that tribal nations have faced until this
point.


\subsection{Barriers to Independence}
An extensive review of the barriers facing tribal energy sovereignty was
published in early 2024 \cite{raimi_securing_2024}. The `key barriers to tribal
energy sovereignty' proposed in this paper can be categorized into two factors:
historical (contexts that cannot be changed) and modern (challenges that could,
in theory, be changed). The barriers titled `Ethnic Cleansing,' `Forced
Migration,' `Forced Coexistence,' and `Land Fragmentation,' are examples of
historical factors; these are challenges that have drastically altered the
identities of native communities as a result of U.S. government actions. The
barriers of `Inadequate Consultation,' `Federal Bureaucracy,' `Lack of
Institutional Capacity,' and `Inability to access tax credits,' are examples of
modern factors that could be addressed with proper intervention (refer to [9]
for further definitions and examples of these factors). Unsurprisingly, the main
modern barriers to tribal energy sovereignty are influenced most heavily by
economic and regulatory/political factors. An important nuance to the complexity
of regulatory solutions to energy sovereignty is highlighted in \cite{kronk2012}
through a quote by Former Senator Ben Campbell: 
\begin{quote}
  The Committee on Indian Affairs has been informed over the year that the
  Secretarial approval process is often so lengthy that outside parties, who
  otherwise would like to partner with Indian tribes to develop their energy
  resources are reluctant to become entangled in the bureaucratic red tape that
  inevitably accompanies the leasing of Tribal resources.
\end{quote}
Thus, this important lesson from previous energy endeavors indicates that
over-regulation can be as detrimental to tribal goals as under-regulation. This
complex problem of the proper level of regulation, coupled with economics and
technology, are major reasons as to why no tribe has achieved complete energy
sovereignty.

%%%%%%%%%%%%%%%%%%%%%%%%%%%%%%%%%%%%%%%%%%%%%%%%%%%%%%%%%%%%%%%%%%%%%%%%%%%%%%%%
\section{Preliminary Results}
The preeminent organization for the use of nuclear energy for the advancement of
tribal nations is the U.S. Department of Energy's Office of Nuclear Energy
Nuclear Energy Tribal Working Group (NETWG). This group, which includes
representation of 12 tribes and tribal organizations, is dedicated to aiding in
the expansion of educational and economic opportunities, management of spent
fuel, emergency preparedness, and advancement of emergent reactor technologies
for tribal groups \cite{department_of_energy_nuclear_2023}. Economic development
and technological advancement are directly related to the topic of energy
sovereignty — while the other initiatives may benefit overall goals of
self-sufficiency, they will not be a core focus of this work. The NETWG has two
publicly available documents that report on the status of energy-related
U.S.–tribal relations. The first, published two years after the NETWG charter,
discusses the history of consent-based siting on tribal land
\cite{nuclear_energy_tribal_working_group_2016}, while the second reports on the
status of educational opportunities in native lands
\cite{nuclear_energy_tribal_working_group_2016}. Additionally, the NETWG
announced a \$1.5 million funding opportunity in 2023 for the facilitation of
increased communications between tribal and federal nuclear energy
representatives. Despite limited publicly available knowledge regarding in-depth
government efforts and initiatives, it is evident that there are groups
interested in and resources available for further work pertaining to
nuclear-based tribal development.

\subsection{Further Investigations}
Due to the significant knowledge and development gaps surrounding this area of
research, this study aims to serve as initial grounds for further exploration
into nuclear energy solutions for native nations. The authors intend to survey
tribal communities to determine the interest in tribal-owned nuclear power, as
well as interview tribal leaders to strengthen context surrounding economic,
regulatory, and cultural (e.g., education, risk-perception) challenges.
Additionally, this study will include a review of existing federal energy and
tribal legislation and nuclear power regulation. These reviews will determine
the extent of under or over-regulation regarding nuclear technology deployment
within tribal nations and may result in recommendations for legal or regulatory
changes. Subsequently, the authors will demonstrate a feasibility utilizing a
tribal-owned microreactor deployment scenario. The authors will subsequently The
progress and findings of this work will be presented at the 2024 American
Nuclear Society Student Conference.


%%%%%%%%%%%%%%%%%%%%%%%%%%%%%%%%%%%%%%%%%%%%%%%%%%%%%%%%%%%%%%%%%%%%%%%%%%%%%%%%


%%%%%%%%%%%%%%%%%%%%%%%%%%%%%%%%%%%%%%%%%%%%%%%%%%%%%%%%%%%%%%%%%%%%%%%%%%%%%%%%
% \section{Acknowledgments}

%%%%%%%%%%%%%%%%%%%%%%%%%%%%%%%%%%%%%%%%%%%%%%%%%%%%%%%%%%%%%%%%%%%%%%%%%%%%%%%%
\bibliographystyle{ans}
\bibliography{2024-fisher-stucon}
\end{document}

